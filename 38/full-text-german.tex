\documentclass[a4paper, 11pt, oneside, german]{article}
\usepackage[utf8]{inputenc}
\usepackage[T1]{fontenc}
\usepackage[ngerman]{babel}
\usepackage{fbb} %Derived from Cardo, provides a Bembo-like font family in otf and pfb format plus LaTeX font support files
\usepackage{booktabs}
\setlength{\emergencystretch}{15pt}
\begin{document}
\begin{titlepage} % Suppresses headers and footers on the title page
	\centering % Centre everything on the title page
	\scshape % Use small caps for all text on the title page

	%------------------------------------------------
	%	Title
	%------------------------------------------------
	
	\rule{\textwidth}{1.6pt}\vspace*{-\baselineskip}\vspace*{2pt} % Thick horizontal rule
	\rule{\textwidth}{0.4pt} % Thin horizontal rule
	
	\vspace{0.75\baselineskip} % Whitespace above the title
	
	{\LARGE Jahreshefte des Vereins für vaterländische Naturkunde in Württemberg\\}
	% Title
	
	\vspace{0.75\baselineskip} % Whitespace below the title
	
	\rule{\textwidth}{0.4pt}\vspace*{-\baselineskip}\vspace{3.2pt} % Thin horizontal rule
	\rule{\textwidth}{1.6pt} % Thick horizontal rule
	
	\vspace{1\baselineskip} % Whitespace after the title block
	
	%------------------------------------------------
	%	Subtitle
	%------------------------------------------------
	
	{\scshape\Large Volume 38\\} % Subtitle or further description
	
	\vspace*{1\baselineskip} % Whitespace under the subtitle
	
    % Subtitle or further description
    
	%------------------------------------------------
	%	Editor(s)
	%------------------------------------------------

    %------------------------------------------------
	%	Cover photo
	%------------------------------------------------
	
	%------------------------------------------------
	%	Publisher
	%------------------------------------------------
		
	\vspace*{\fill}% Whitespace under the publisher logo
	
	 % Publication year
	
	{\small 1$^{st}$ Edition, W"urttemberg 1882} % Publisher

	\vspace{1\baselineskip} % Whitespace under the publisher logo

    Internet Archive Online Edition  % Publication year
	
	{\small Attribution NonCommercial ShareAlike 4.0 International } % Publisher
\end{titlepage}
\setlength{\parskip}{1mm plus1mm minus1mm}
\setcounter{tocdepth}{2}
\setcounter{secnumdepth}{3}
\tableofcontents
\clearpage
\section{Bericht "uber zwei Gallertmeteoritenf"alle}
\paragraph{}
Nach den bisherigen Beobachtungen sind 8 F"alle von Gallertmeteoriten zu verzeichnen, worunter einer vom Jahr 1828 oder 1829 (von Allport Derbyshire), dessen Analyse 32,00 Schwefel, 34,09 Eisenoxyd, 43,59 Kohle und spezifisches Gewicht 2 ergab. Die "ubrigen F"alle kamen nicht zur Untersuchung, sind aber von durchaus glaubw"urdigen Personen bezeugt. Alle sind zusammengestellt von [Georg Heinrich von] Boguslawski, aus welchem wieder Dr. Otto Ule (\emph{Die Wunder der Sternenwelt} [sic], 2$^{nd}$. Ausgabe von D. Klein S. 360 ff.) die wichtigsten erw"ahnten 8 F"alle beschrieben hat.

Die F"alle gleichen sich unter einander und erregen, sobald sie als voll erwiesen angenommen werden k"onnen, das h"ochste Interesse schon wegen der merkw"urdigen chemischen Zusammensetzung. Bis jetzt scheint aber die Wissenschaft doch eine absolute Gewissheit nicht zuzulassen. Ich erachte es daher f"ur geboten, zwei weitere ebenfalls blo"s h"ochstwahrscheinlich gemachte F"alle mitzuteilen, zwei F"alle, welche mit den von v. Boguslawski beschriebenen v"ollig "ubereinstimmen, wovon "uberdies der eine noch eine besondere Merkw"urdigkeit f"ur uns h"atte: denn er w"are der einzige in W"urttemberg beobachtete Meteorfall.

Beide F"alle sind von Personen berichtet, denen die Beobachtungsf"ahigkeit nicht abgesprochen werden kann: deren Charakter auch daf"ur b"urgt, dass sie die Wahrheit sagen wollten.

Die Wahrscheinlichkeit ihrer Aussagen aber wird zur Gewissheit eben durch den Inhalt der Mitteilung selbst. Beide Gew"ahrsm"anner sind nicht Sachverst"andige, sie wussten blo"s, dass ich mich f"ur Meteoriten sehr interessiere. H"atten sie mir etwa einen B"aren aufbinden wollen, so w"urden sie, wie jener Falsificator des Eisfuchses zum n"achsten besten Bilderbuch griff, so entweder nach bisher ging und g"aben Beschreibungen erz"ahlt, oder zu einem Buche gegriffen, und hiernach von einer Feuererscheinung, daher sicher von Lichterscheinung, Kanonenschuss, Kleingewehrgeknatter als begleitenden Erscheinungen erz"ahlt haben. Es ist im h"ochsten Grade unwahrscheinlich, dass sie eine Beschreibung gerade des seltensten Vorgangs gefunden oder herausgenommen h"atten.

In den gew"ohnlichen Lehr- und Leseb"uchern findet sich aber nichts von Gallertmeteoriten. Diese Mitteilungen geschahen "uberdies so, dass auch eine Vorbereitung darauf kaum m"oglich gewesen w"are.

So nehme ich an, dass sie Beobachtetes treu erz"ahlt haben: ist aber wahr, was sie erz"ahlen, so ist ein Zweifel dar"uber, dass man es wirklich mit kosmischen Massen zu tun hat, ausgeschlossen.

Es wurden allerdings schon Gallertmassen gefunden, welche man sich nicht sofort erkl"aren konnte, weshalb man zur Erkl"arung aus Sternschnuppen griff. Allein sie waren nach aller Beschreibung doch ganz anderer Art, strukturlose Schleimmassen; sie wurden bei genauer Untersuchung als tellurische Substanzen erkannt. Auf solch gefundene Dinge l"asst sich bei dem heutigen Stand der Wissenschaft ein Wert nicht legen. Bei den von mir zu berichtenden F"allen ist durch die Art der Form jeder Gedanke an terrestrische Entstehung, z. B. Froschlaich, Nostocalgen von vornherein ausgeschlossen: es sind alle begleitende Umst"ande angegeben und die Identit"at des Gefundenen oder richtiger beobachteten mit dem wirklich gefallenen Gegenstand au"ser Zweifel. Eine Erkl"arung f"ur eine solche Masse auf einem trocknen Grasboden oder einem Baume, wie ich dies nun n"aher mitteilen werde, gibt es nicht.

Die zwei F"alle erfolgten beide h"ochstwahrscheinlich im Jahre 1848 und zwar in einer Nacht zwischen 9. bis 14. August.

Der erste meiner Gew"ahrsm"anner, Herr Mechaniker M"uller in Reutlingen, gab mir an:

Ich bin im August 1837 geboren. Mein Vater war Kgl. bayrischer Bezirksgeometer in Ottobeuren bei Memmingen, vorher Rechnungs-Revisor bei der Landesvermessung in Bayern.

Ich besuchte die ausgezeichnete Volksschule in Ottobeuren, war stets der Erste meiner Klasse und besitze noch die ersten Preise, welche ich w"ahrend meines Schulbesuchs bis zum 14. Jahre jedes Jahr erhielt. In dem Unterricht schon erfuhren wir von Meteorsteinen und Sternschnuppen. Mein Vater war mit den betreffenden Naturvorg"angen vertraut und setzte sie uns in der Unterhaltung auseinander.

Ich wei"s nicht mehr, war es im Jahr 1848 oder 1849, ich glaube 1848, nach der Heuernte, als ich an einem Sommerabend (bei anhaltend trockenem Wetter) den ganz klaren Himmel beobachtete, und hier eine Unzahl von Sternschnuppen fallen sah. Sie fielen in allen Richtungen. Der Himmel bildete ein Netz von Lichtstreifen, kreuz und quer. Einzelne sah ich in einem Winkel sich bewegen, als ob sie von einem Gegenstand in der Atmosph"are abgesprungen w"aren. Dabei stoben viele garbenartig auseinander. Diesem Ph"anomen sah ich in der N"ahe des Orts zu und, um es vollst"andig beobachten zu k"onnen, begab ich mich auf eine kleine Anh"ohe, 1/4 Stunde vom Ort gelegen, wo ich gegen Osten einen Tannenwald in der Entfernung von 1/4 Stunde unter mir hatte. Der H"ugel war etwa so hoch als die Tannengipfel, ich stand also in gleicher H"ohe mit den Tannenspitzen.

Ich bemerke, dass es lange trockenes Wetter war; das Grundst"uck, auf welchem ich stand, bildete einen H"ugel, es war eine Wiese (mit vielen Spr"ungen in Folge der Trockenheit), ihr Untergrund war ein Diluvialger"olle, wie in der ganzen Gegend und darauf eine etwa 1 1/2 tiefe Sandschichte, welche eine Decke von Graspflanzen trug.

Pl"otzlich h"orte ich "uber den Wald her (hinter mir) Zischen und Sausen und nachdem ich es 2-3 Sekunden geh"ort, einen Fall hinter mir in der Entfernung von 3-4 Schritten. Der Ton wies nach Osten, nach dem Wald hin. Beim Auffallen tat es, wie wenn eine mit Luft gef"ullte Ochsenblase zersprengt w"urde. W"ahrend dessen h"orte ich auch noch das Echo des Zischtons vom Walde her.

Ich drehte mich rasch um und suchte den offenbar gefallenen Gegenstand. Da ich aber einen solchen wegen der Dunkelheit nicht sofort unterscheiden konnte, so bezeichnete ich die Stelle, wo der Fall stattgefunden haben musste, mit meinem Stocke, um am andern Tag dieselbe wieder aufzusuchen. Das Grundst"uck war unser Eigentum; ich h"atte sie auch ohne diese Bezeichnung leicht wieder gefunden; denn sie war an der Grenze in einem Winkel, welchen diese mit dem Nachbargrundst"uck machte, also schon gen"ugend f"ur mich bezeichnet.

In aller Fr"uhe des andern Tages begab ich mich auf den Platz und fand wirklich nur 1 m von der durch den Stock bezeichneten Stelle eine gallertartige Kugel, bestehend wieder aus Kugeln von Rostfarbe. Sie lag auf dem d"unnen Grase; unten hatte sich eine Fl"ache plattgedr"uckt.

Der Durchmesser der Kugel war 30 cm.

Ich stie"s die Kugel mit dem Fu"se an, sie kam in zitternde Bewegung, welche sich l"angere Zeit fortsetzte, bis die Masse nach immer schw"acheren Schwingungen zum Ruhepunkt zur"uckkehrte. Nun ber"uhrte ich sie, sie war klebrig, es blieb aber von der Substanz nichts an den Fingern h"angen, auch ver"anderte sich weder vom Sto"s des Fu"ses noch von der Ber"uhrung des Fingers ihre Form.

Nun komme ich an die n"ahere Beschreibung der Kugel. Sie bestand aus rostfarbenen runden K"orpern, welche an einander lagen, ohne dass ich ganz scharfe Grenzlinien wahrgenommen h"atte.

Eine dieser Kugeln war 4 cm gro"s, die meisten kleiner bis zu 2 cm.\footnote{Diese Tatsache, wie sie auch im zweiten Fall beobachtet wurde, schlie"st jede Verwechslung mit Froschlaich aus.} Im Innern der Kugeln bemerkte ich fadenf"ormige Linien von dunkler, sogar von schwarzer Farbe; die Zwischenr"aume zwischen den Kugeln (denn diese legten sich nicht unmittelbar an einander) waren von dunkler Masse ausgef"ullt und hatten ebenfalls eine Art Struktur, wie ich sie an den Kugeln deutlich beobachtete. Das Ganze hatte ungef"ahr das Aussehen eines dunkelgelben Froschlaich-Klumpens, mit F"aden durchzogen, wodurch sie sich eben, sowie durch die verschiedene Gr"o"se der Kugeln ganz unzweifelhaft von Froschlaich unterschied. Die Oberfl"ache war schwach durchsichtig, so dass ich die Form der Kugeln als runde und birnf"ormige K"orper erkennen konnte.

Ich wusste nichts mit dem Ding zu tun. Der Klumpen erregte Ekel in mir. Ich lie"s ihn liegen. Aber ich hatte keinen andern Glauben, als dass er ein Meteor sei. Ich habe n"amlich nichts der Art je auf der Erde, ebenso nichts vor- und nachher auf der Wiese gesehen. Auch meine Geschwister, welche sie sahen, erinnerten sich nie, solche Gebilde dort gesehen zu haben.

Nach einigen Tagen war nichts mehr davon zu sehen.

Einen Eindruck im Boden hinterlie"s sie nicht. Ich habe die Kugel nicht umgekehrt, sie blieb also auf der Erde liegen, wo sie nach und nach eingetrocknet sein muss.

Zum Schlusse bemerke ich, dass ich, weil ich blo"s von Meteorsteinen wusste, am Morgen den Platz auf ziemlich weitem Umkreis auch nach einem Stein absuchte, aber keinen, auch kein Loch fand, weshalb ich sicher war, dass nichts anderes als der Klumpen der am Abend vorher gefallene Gegenstand war.

W"ahrend des Falls beobachtete ich weder einen Knall oder Knattern, noch eine den Fall begleitende Lichterscheinung, welche sich, wenigstens wenn sie stark gewesen, mir doch von hinten her durch den Wiederschein h"atte bemerkbar machen m"ussen.

Die Fallrichtung ging "uber den Wald her von Osten nach Westen, was ich sogleich, wie nachher noch in dem Echo des Waldes erkennen konnte.

Au"ser meiner Familie machte ich keine Mitteilung. Meine Eltern und Br"uder sind gestorben.

Dies die Erz"ahlung des Hergangs Seitens der Herrn M"uller, welcher diese Angaben mit seinem Ehrenwort bekr"aftigt.

Wahrscheinlich in derselben Nacht war es, dass eine Kameradschaft Bauernburschen in Neuhausen, O. A. Urach, mit einem K"ubel Wein auf ein nahes Baumgut sich begaben, um denselben dort zu trinken. Unter ihnen war der jetzt als Kaufmann in Linsenhofen, O. A. N"urtingen, ans"assige Kaufmann Gottfried Seiz, mein Gew"ahrsmann, damals 20 Jahre alt. — Ich bemerke, dass den ledigen Bauernjungen der Wirtshausbesuch untersagt war, dass sie sich also auf solche Art hierf"ur schadlos hielten. Einer um den andern musste den Abend mit der Weinlieferung aushalten. G. Seiz versichert, dass der Anteil eines Jeden nicht zu gro"s gewesen, und dass es stets ohne Betrunkene abgegangen sei, und an jenem Abend auch alle n"uchtern gewesen seien.

So sa"s, bei schon eingetretener Dunkelheit, die Kameradschaft unter einem gro"sen Birnbaum und hatte noch Zeit genug "ubrig, das prachtvolle Schauspiel des au"serordentlichen Sternschnuppenfalls zu beobachten. Pl"otzlich zog ein leuchtender Gegenstand "uber den Berg von Osten (von Neuffen her), senkte sich herab und fiel in die Krone des Birnbaumes, unter dem die Leute sa"sen, um dort in hunderte von leuchtenden Funken auseinander zu stieben. Unmittelbar nach dem Aufprallen im Gipfel h"orte man K"orper durch die Zweige und Bl"atter fallen, einer fiel in den K"ubel. Der K"ubel war fast leer: er wurde vollends geleert und nun fand sich auf dem Boden ein aus gr"unlichen schleimigen Kugeln bestehender Klumpen, welchen Seiz herausnahm und n"aher betrachtete.

Eine Kugel hatte Haselnussgr"o"se, die anderen waren kleiner bis zur Stecknadelkopfgro"se. Die Kugeln selbst waren von Linien oder F"aden durchzogen, was Seiz ganz genau beobachtete. Die Masse wurde nicht aufbewahrt und ging so verloren.

Aus diesen Beschreibungen ist die vollst"andige Übereinstimmung mit den von Boguslawski berichteten F"allen zu entnehmen, wie denn auch beide von mir berichtete F"alle unter sich (bis auf die Farbe und die Lichterscheinung, welche aber, wenn sie eine schwache war, dem Beobachter doch entgehen, umso leichter entgehen konnte, als die vielen F"alle schon Licht verbreiteten) "ubereinstimmen.

Ich m"ochte an diese Mitteilung die Bitte kn"upfen, auf diesen Spuren weiter zu forschen und in vorkommenden F"allen f"ur sorgsamste Erhaltung der Massen zu sorgen. Alle Berichte sprechen daf"ur, dass dieselben sehr schnell vertrocknen oder gar verdunsten. Es m"usste also, soll die Masse erhalten bleiben, hiergegen durch luftdichten Verschluss Vorsorge getroffen werden.

Vielleicht gelingt es auch unsern Mitgliedern, durch Nachfrage weitere F"alle zu erheben. Ich w"are f"ur Mitteilungen dar"uber sehr verbunden.

Ausgeschlossen m"ussten nat"urlich solche sein, wo das Fallen der Massen selbst nicht festgestellt werden k"onnte. Soviel ist sicher: die Wissenschaft ist aufgefordert, diese Anf"ange zu verfolgen und bei der n"achsten besten Gelegenheit diese im h"ochsten Grade wichtige Frage durch genaue Feststellung der Tatsachen und Untersuchung des Gegenstands zur Entscheidung zu bringen.
\clearpage
\end{document}

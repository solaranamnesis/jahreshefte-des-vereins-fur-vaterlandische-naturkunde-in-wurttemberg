\documentclass[a4paper, 12pt, oneside]{article}
\usepackage[utf8]{inputenc}
\usepackage{fouriernc}
\usepackage{booktabs}
\setlength{\emergencystretch}{15pt}
\usepackage{float}
\begin{document}
\begin{titlepage} % Suppresses headers and footers on the title page
	\centering % Centre everything on the title page
	\scshape % Use small caps for all text on the title page

	%------------------------------------------------
	%	Title
	%------------------------------------------------
	
	\rule{\textwidth}{1.6pt}\vspace*{-\baselineskip}\vspace*{2pt} % Thick horizontal rule
	\rule{\textwidth}{0.4pt} % Thin horizontal rule
	
	\vspace{0.75\baselineskip} % Whitespace above the title
	
	{\LARGE Annual Bulletin of the Association for National Natural History in Württemberg\\}
	% Title
	
	\vspace{0.75\baselineskip} % Whitespace below the title
	
	\rule{\textwidth}{0.4pt}\vspace*{-\baselineskip}\vspace{3.2pt} % Thin horizontal rule
	\rule{\textwidth}{1.6pt} % Thick horizontal rule
	
	\vspace{1\baselineskip} % Whitespace after the title block
	
	%------------------------------------------------
	%	Subtitle
	%------------------------------------------------
	
	{\scshape\Large Volume 38\\} % Subtitle or further description
	
	\vspace*{1\baselineskip} % Whitespace under the subtitle
	
    % Subtitle or further description
    
	%------------------------------------------------
	%	Editor(s)
	%------------------------------------------------

    %------------------------------------------------
	%	Cover photo
	%------------------------------------------------
	
	%------------------------------------------------
	%	Publisher
	%------------------------------------------------
		
	\vspace*{\fill}% Whitespace under the publisher logo
	
	 % Publication year
	
	{\small 1$^{st}$ Edition, Württemberg 1882} % Publisher

	\vspace{1\baselineskip} % Whitespace under the publisher logo

    Internet Archive Online Edition  % Publication year
	
	{\small Attribution NonCommercial ShareAlike 4.0 International } % Publisher
\end{titlepage}
\setlength{\parskip}{1mm plus1mm minus1mm}
\clearpage
\tableofcontents
\clearpage
\section{Report on Two Gelatin Meteorite Falls}
\paragraph{}
According to previous observations there are eight falls of gelatin meteorites on record, among them is one from the year 1828 or 1829 (from Allport Derbyshire) whose analysis revealed 32.00\% sulfur, 34.09\% iron oxide, and 43.59\% coal and specific gravity 2. The remaining cases did not get an investigation, however they were witnessed by quite credible people. All are compiled by [Georg Heinrich von] Boguslawski, from which Dr. Otto Ule (\emph{The Wonder of the Starry World} [sic], $2^{nd}$ Edition with D. Klein, page 360 and the following pages) has again described the most important eight cases mentioned.

The falls are similar to each other and, as soon as they can be supposed as completely proven, excite the highest interest because of their peculiar chemical composition. Up till now, however, science has not admitted absolute certainty. I therefore consider it necessary to also communicate just two additional, more likely than not, falls; two falls which totally correspond with Boguslawski and one which furthermore has a special strangeness: for it is the only observed meteorite fall in Württemberg.

Both falls are reported by persons who cannot be denied their observational ability: their character also vouches in favor of their desire to tell the truth.

The plausibility of their statements, however, becomes certain simply through the content of the message itself. Both informants are not technical experts, they knew only that I was very interested in meteorites. If they had wanted to pull my leg they would, like that deceiving arctic fox grabbing the best nearby storybook, either narrate a description in accordance with what had hitherto been given, or resort to a book, and after this tell of a fire phenomenon, from there of a light phenomenon, cannon shots, small rifle crackling as accompanying the phenomenon. It is, in the highest degree, unlikely that they would have found or extracted a genuine description of this uncommon event.

In the standard textbooks and guidebooks, however, nothing is found on gelatin meteorites. Furthermore, these statements were made in such a way that preparation could hardly be possible.

So, I assume that the observed was faithfully recounted: but if it is true what they tell, then it is out of question and beyond doubt that one is really dealing with cosmic masses.

It was certainly, however, gelatin masses that were found, which no one was able to immediately explain, for which reason one resorted to an explanation from shooting stars. Only they had a description of an entirely different kind, structureless slime masses; they were identified from detailed investigation as telluric substances. Importance cannot be attached to such stated things in the current state of science. In my reporting of the falls, based on the nature of their form, any thought of terrestrial origin, i.e. frogspawn, \emph{Nostoc} algae, is from the outset out of question: all accompanying circumstances are stated and the identity of the discovered, or more correctly, the observations of a truly fallen object are without doubt. An explanation in favor of such a mass upon dry grassy ground or a tree, as I am about to now communicate, does not exist.

The two falls most likely took place in the year 1848, and in fact, between the nights of the 9th and 14th of August.

The first source of mine, Mr. Müller, a mechanic in Reutlingen told me:

I was born in August 1837. My father was royal Bavarian district geometer in Ottobeuren near Memmingen, formerly accounting auditor of land surveying in Bavaria.

I attended the excellent primary school in Ottobeuren, was always first in my class and still own the first prizes, which I received during my school attendance each year, until the 14th year. In the classes we certainly learned about meteorites and shooting stars. My father, apart from being conversant with the relevant natural processes, told us about them in conversation.

I can no longer recall, in the year 1848 or 1849, I believe 1848, after the hay harvest, I was observing a very clear sky one summer evening (when the weather was still dry) and at this point saw a host of shooting stars fall. They fell in all directions. The sky became a web of crisscrossing streaks of light. I saw some individuals move at an angle, as if they had bounced off an object in the atmosphere. In the course of this many bursts of fire sprayed forth. I witnessed this phenomenon from the vicinity of this location and, in order to completely observe it, went to a small mound, a quarter of an hour from the located spot, where to the east I had seen a pine forest in the distance about a quarter of an hour below me. The hill was about as high as the peaks of the pines, I thus stood at an equal height with the pine tops.

I note that the weather had been dry for a long time; the plot of land on which I stood formed a hill, it was a meadow (with many cracks as a consequence of the dryness), the subsoil was a diluvial scree common to the whole area and upon that was an approximately one and a half feet deep layer of sand, which bore a covering of grass plants.

All of a sudden I heard hissing and whizzing across the forest (behind me) and, after listening it for two to three seconds, a fall behind me in the distance about three to four paces away. The sound was directed east, hinting towards the forest. When this action caught my attention, it was like an air-filled ox bladder had burst. Through this I heard the echo of the hissing from the forest.

I quickly turned and searched for the apparently fallen object. But since I was unable to immediately distinguish any such thing due to the darkness, I marked the spot where the fall should have occured with my stick, so that I could visit it the next day. The plot of land was our property; I would have also easily found it again without this marking because it was at the corner of the boundary that it made with the adjacent property, hence it was already sufficiently marked for me.

Bright and early the next day I went to the place and actually found, only one meter from the spot marked by the stick, a gelatin-like ball, likewise consisting of rust-colored balls. It was located on thin grass; a surface had flattened out beneath it.

The diameter of the ball was thirty centimeters.

I kicked the ball with my foot, it began a trembling motion which continued for a long time until the mass returned to its resting point with ever weakening vibrations. Now I touched it, it was sticky, but nothing of the substance stuck to the fingers, nor did its form change from the kick of the foot or the touch of the fingers.

I now come to the detailed description of the ball. It consisted of rust-colored round bodies, which lay against each other, free of any noticable sharp boundary lines.

One of these balls was four centimeters in size, the majority smaller up to two centimeters.\footnote{This fact, as was also observed in the second fall, excludes any confusion with frogspawn.} In the interior of the balls I noticed filamentous lines of dark, even black, color; the interstices between the balls (because they did not lie directly on each other) were filled with a dark mass and likewise had a kind of structure, such as I distinctly observed on the balls. The whole thing had roughly the appearance of a dark yellow frogspawn clump, pervaded with filaments, and just by this, plus the various sizes of the balls, made it entirely and undoubtedly different from frogspawn. The surface was slightly transparent, so that I could recognize the form of the balls as round and pear-shaped bodies.

I didn't know what to do with the thing. The clump arouse disgust in me. I left it laying there. However, I had no change of thought, that it was a meteor. I have never, you see, seen anything of this kind on the Earth, nor have I seen anything before or after in the meadow. Even my siblings, who saw it, never remembered seeing such a thing.

After a few days, nothing more of it was visible.

It left no effect on the ground. I did not turn the ball over, so it remained laying on the ground while it must have dried up little by little.

In conclusion I note that, because I knew of meteorites, in the morning I also searched the place for a stone within a fairly extensive perimeter, but nothing, not even a hole was found, for which reason I was certain that nothing besides the clump had fallen the evening before.

During the fall I observed neither a bang or rattle, nor an appearance of light accompanying the fall, which, at least if it was strong, should have been noticeable by me from behind through a reflection.

The direction of the fall was over the forest from east to west, which I immediately realized, just as later on, in the echo of the forest.

Except for my family, I made no disclosure. My parents and brothers have passed away.

This is the narrative of the course of events by Mr. Müller, who confirms these statements with his word of honor.

Probably on the same night, a camaraderie of country boys in Neuhausen, Urach County, went with a pail of wine to a nearby tree to imbibe. Among them was Gottfried Seiz, my source, now a merchant in Linsenhofen, Nürtingen County, who was at the time twenty years old. — I note that unmarried country boys were forbidden from the tavern and therefore were considered harmless in this manner. One among the others had to maintain the evening wine supply. G. Seiz insists that everyone's share was not too great, and that it always went off without drunks and everyone that evening was sober.

With darkness already materializing, the camaraderie sat under a large pear tree with still enough time left to observe the magnificent spectacle of the extraordinary shooting stars. All of a sudden a luminous object moved over the mountain from the east (from Neuffen), descended and fell into the crown of the pear tree under which the guys sat, therein spraying forth hundreds of bright sparks. Immediately after impacting the top, bodies were heard falling through the branches and leaves, one fell into the pail. The pail was nearly empty: it was then completely emptied and there was now found on the ground clumps consisting of viridescent slimy balls, which Seiz picked out and looked at closely.

One ball was the size of a hazelnut, the others were smaller, all the way down to the size of a pinhead. The balls themselves were streaked with lines or filaments, which Seiz observed quite closely. The mass was not preserved and was therefore lost.

From these accounts a full agreement with the falls reported by Boguslawski can be inferred, as well as both the falls reported by me which also coincide (apart from the color and the light phenomenon, which, however, if it was faint would elude the observer, all the more easier to miss since many falls were certainly radiating light).

I would like to include with this communication a request for further research on these clues and for ensuring the most careful preservation of the masses in the falls that are found. All reports suggest that they rapidly wither or evaporate quite quickly. Thus, if the mass is to remain preserved, a hermetically sealed provision should be made to prevent this.

Perhaps even our members will be successful by way of making inquiries into additional falls. I would be very obliged for such information.

Naturally, those where the falling of masses themselves cannot be ascertained need to be excluded. This much is certain: science is challenged to pursue these beginnings and, with the next best opportunity, to bring this highly important matter into judgement through precise assessment of the facts and analysis of the object.
\clearpage
\end{document}
